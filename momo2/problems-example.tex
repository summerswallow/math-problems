%Example of use of oxmathproblems latex class for problem sheets
\documentclass{oxmathproblems}


%(un)comment this line to enable/disable output of any solutions in the file
%\printanswers

%define the page header/title info

\setlength{\parindent}{0pt}
\begin{document}

\section*{L'H\^opital's Rule}
L'Hopital's rule is used to calculate limits.

If $\displaystyle{\lim_{x\to a} f(x)\to\infty}$ and $\displaystyle{\lim_{x\to a} g(x)\to\infty}$ OR $\displaystyle{\lim_{x\to a} f(x)=0}$ and $\displaystyle{\lim_{x\to a} g(x)=0}$   then
$$\lim_{x\to a} \frac{f(x)}{g(x)} = \lim_{x\to a} \frac{f'(x)}{g'(x)}$$ 

\begin{questions}

\question
Calculate the following limits:
\begin{parts}
  \part $\displaystyle{\lim_{z\to 0} \frac{\sin(z)}{2z}}$
  \vspace{1.5in}
  
  \part $\displaystyle{\lim_{y\to \infty} \frac{\ln(y)}{3y}}$
  
  \part $\displaystyle{\lim_{x\to 1} \frac{x^3-3x^2+2}{x^3-x^2-x+1}}$
  
  \part $\displaystyle{\lim_{x\to 3} \frac{3x^2+1}{x+4}}$
  
  \part $\displaystyle{\lim_{x\to 3} 6x}$
  
\end{parts}

\begin{solution}
  The solution would go here
\end{solution}

\end{questions}
\newpage

\section*{Limits of towards infinity and zero}

The limit function as a variable tends towards infinity is determined by what happens to the function for large values of the variable.

Consider $\displaystyle{\lim_{x\to \infty} \frac{1}{x}}$. If $x=10$ then $\frac{1}{x}=\frac{1}{10}$, if $x=100$ then $\frac{1}{x}=\frac{1}{100}$, if $x=1000$ then $\frac{1}{x}=\frac{1}{1000}$, if $x=10000$ then $\frac{1}{x}=\frac{1}{10000}$, and so forth. So if $x$ gets large $\frac{1}{x}$ gets small. In fact when $x$ gets infinitely large $\frac{1}{x}$ gets infinitely small. And in fact, $$\lim_{x\to\infty} \frac{1}{x}=0$$

Conversely if $x$ gets small $\frac{1}{x}$ gets large and in fact,
$$\lim_{x\to 0} \frac{1}{x}\to\infty$$

But you must be wary of the sign of the limit, for example,
$$\lim_{x\to 0} -\frac{1}{x}\to-\infty$$

\begin{questions}
\question Calculate$\displaystyle{\lim_{z\to \infty} -z^2}$
\vspace{2in}
\question
Calculate $\displaystyle{\lim_{y\to \infty} 10^y}$
\vspace{2in}
\question
Calculate $\displaystyle{lim_{x\to \infty} 10^{-x}}$
\vspace{2in}
\question{(Bonus)}
Calculate $\displaystyle{\lim_{x\to 0} e^{-{\frac{1}{x}}}$
\end{questions}
\end{document}
