\section*{Limits of towards infinity and zero}

The limit function as a variable tends towards infinity is determined by what happens to the function for large values of the variable.

Consider $\displaystyle{\lim_{x\to \infty} \frac{1}{x}}$. If $x=10$ then $\frac{1}{x}=\frac{1}{10}$, if $x=100$ then $\frac{1}{x}=\frac{1}{100}$, if $x=1000$ then $\frac{1}{x}=\frac{1}{1000}$, if $x=10000$ then $\frac{1}{x}=\frac{1}{10000}$, and so forth. So if $x$ gets large $\frac{1}{x}$ gets small. In fact when $x$ gets infinitely large $\frac{1}{x}$ gets infinitely small. And in fact, $$\lim_{x\to\infty} \frac{1}{x}=0$$

Conversely if $x$ gets small $\frac{1}{x}$ gets large and in fact,
$$\lim_{x\to 0} \frac{1}{x}\to\infty$$

But you must be wary of the sign of the limit, for example,
$$\lim_{x\to 0} -\frac{1}{x}\to-\infty$$

\begin{questions}
\question Calculate$\displaystyle{\lim_{z\to \infty} -z^2}$
\begin{solution}[1.5in]
  As $z$ gets large $-z^2$ becomes a negative number with large magnitude. (Try with some integers $1,10, 100, 1000,\ldots$ become $-1, -100, -10000, -1000000,\ldots$. Therefore,
  $$\lim_{z\to \infty} -z^2 \mapsto -\infty$$
\end{solution}
\question
Calculate $\displaystyle{\lim_{y\to \infty} 10^y}$
\begin{solution}[1.5in]
  As $y$ gets large $10^y$ becomes a large. (Try with some integers $1, 2, 3, 4\ldots$ become $10, 100, 1000, 10000,\ldots$. Therefore,
$$\lim_{y\to \infty} 10^y\mapsto \infty$$
\end{solution}
\question
Calculate $\displaystyle{\lim_{x\to \infty} 10^{-x}}$
\begin{solution}[1.5in]
  Two ways to view this, as $x$ gets large $10^{-x}$ becomes a small. (Try with some integers $1, 2, 3, 4\ldots$ become $.1, .01, .001, .0001,\ldots$. Therefore,
  $$\lim_{x\to \infty} 10^{-x}=0$$
  The other way to see this is that $\displaystyle{10^{-x}=\frac{1}{10^x}}$ and we know $10^x$ gets large as $x$ gets large therefore $\frac{1}{10^x}$ get small.
\end{solution}
\question{(Bonus)}
Calculate $\displaystyle{\lim_{x\to 0} e^{-{\frac{1}{x}}}}$
\end{questions}
