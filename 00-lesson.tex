\section*{Introduction to L'H\^opital's Rule}

\lhopital's rule is used to calculate limits.

\vspace{\baselineskip}
\noindent If $\displaystyle{\lim_{x\to a} f(x)\to\infty}$ and $\displaystyle{\lim_{x\to a} g(x)\to\infty}$ OR $\displaystyle{\lim_{x\to a} f(x)=0}$ and $\displaystyle{\lim_{x\to a} g(x)=0}$   then
$$\lim_{x\to a} \frac{f(x)}{g(x)} = \lim_{x\to a} \frac{f'(x)}{g'(x)}$$ 

\printanswers
\begin{questions}
\question
Calculate
$\displaystyle{\lim_{z\to 0} \frac{\sin(z)}{2z}}$
\begin{solution}[1.5in]
  This is a limit of the form $\frac{\infty}{\infty}$ so \lhopital's rule applies.
  $$\lim_{z\to 0} \frac{\sin(z)}{2z} = \lim_{z\to 0} \frac{\cos(z)}{2} = \frac{1}{2}$$
\end{solution}
\question
Calculate $\displaystyle{\lim_{y\to \infty} \frac{\ln(y)}{3y}}$
\begin{solution}[1.5in]
  This is a limit of the form $\frac{\infty}{\infty}$ so \lhopital's rule applies.
  $$\lim_{y\to \infty} \frac{\ln(y)}{3y}=\lim_{y\to \infty} \frac{\frac{1}{y}}{3} = \lim_{y\to \infty} \frac{1}{3y} = 0$$
\end{solution}
\question
Calculate  $\displaystyle{\lim_{x\to 1} \frac{x^3-3x^2+2}{x^3-x^2-x+1}}$
\begin{solution}[1.5in]
  This is a limit of the form $\frac{0}{0}$ so \lhopital's rule applies.
  $$\lim_{x\to 1} \frac{x^3-3x^2+2}{x^3-x^2-x+1} = \lim_{x\to 1} \frac{3x^2-6x}{3x^2-2x-1} $$
  This is still of the form $\frac{0}{0}$, so apply \lhopital's rules again.
  $$\lim_{x\to 1} \frac{3x^2-6x}{3x^2-2x-1} = \lim_{x\to 1} \frac{6x-6}{6x-2} = 0 $$
\end{solution}
\question
\begin{parts}
    \part Calculate $\displaystyle{\lim_{x\to 3} \frac{3x^2+1}{x+4}}$
    \begin{solution}[1in]
      This limit is {\bf not} of a form you can use \lhopital's rule.
      $$\lim_{x\to 3} \frac{3x^2+1}{x+4} = \frac{28}{7}=4$$
    \end{solution}
    \part Calculate  $\displaystyle{\lim_{x\to 3} 6x}$
    \begin{solution}[1in]
      This limit is {\bf not} of a form you can use \lhopital's rule.
      $$\lim_{x\to 3} 6x = 18$$
      If we took the derivative of the denominator and the numerator of the limit of part (a). We would get
      $$\frac{6x}{1}=6x$$
      This shows that \lhopital's rule {\bf only} applies if the limits are of these special forms
    \end{solution}
\end{parts}
\end{questions}
