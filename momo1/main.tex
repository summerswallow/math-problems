\documentclass[a4paper,twoside,11pt]{article}
\usepackage{a4wide,graphicx,fancyhdr,clrscode,tabularx,amsmath,amssymb,color,enumitem}
\usepackage{algo}

%----------------------- Macros and Definitions --------------------------

\setlength\headheight{20pt}
\addtolength\topmargin{-10pt}
\addtolength\footskip{20pt}

\fancypagestyle{plain}{%
\fancyhf{}

\fancyfoot[LO,RE]{\sffamily Momo Math}
\fancyfoot[RO,LE]{\sffamily\bfseries\thepage}
\renewcommand{\headrulewidth}{0pt}
\renewcommand{\footrulewidth}{0pt}
}

\pagestyle{fancy}
\fancyhf{}

\fancyfoot[LO,RE]{\sffamily Momo Math}
\fancyfoot[RO,LE]{\sffamily\bfseries\thepage}
\renewcommand{\headrulewidth}{1pt}
\renewcommand{\footrulewidth}{0pt}
 \newenvironment{problem}{\begin{center}
    \begin{tabular}{|p{0.9\textwidth}|}
    \hline\\
    }
    { 
    \\\\\hline
    \end{tabular} 
    \end{center}
    }
\\setlength{\parindent}{0pt}
\begin{document}


\section*{Limits of towards infinity and zero}

The limit function as a variable tends towards infinity is determined by what happens to the function for large values of the variable.

Consider $\displaystyle{\lim_{x\to \infty} \frac{1}{x}}$. If $x=10$ then $\frac{1}{x}=\frac{1}{10}$, if $x=100$ then $\frac{1}{x}=\frac{1}{100}$, if $x=1000$ then $\frac{1}{x}=\frac{1}{1000}$, if $x=10000$ then $\frac{1}{x}=\frac{1}{10000}$, and so forth. So if $x$ gets large $\frac{1}{x}$ gets small. In fact when $x$ gets infinitely large $\frac{1}{x}$ gets infinitely small. And in fact, $$\lim_{x\to\infty} \frac{1}{x}=0$$

Conversely if $x$ gets small $\frac{1}{x}$ gets large and in fact,
$$\lim_{x\to 0} \frac{1}{x}\to\infty$$

But you must be wary of the sign of the limit, for example,
$$\lim_{x\to 0} -\frac{1}{x}\to-\infty$$

\begin{problem}{1}
Calculate$\displaystyle{\lim_{z\to \infty} -z^2}$
\end{problem}
\vspace{2in}
\begin{problem}{2}
Calculate $\displaystyle{\lim_{y\to \infty} 10^y}$
\end{problem}
\vspace{2in}
\begin{problem}{3}
Calculate $\displaystyle{lim_{x\to \infty} 10^{-x}}$
\end{problem}

\vspace{2in}
\begin{problem}{4} (Bonus)
Calculate $\displaystyle{\lim_{x\to 0} e^{-{\frac{1}{x}}}$
\end{problem}


Note: This is by no means a full and complete introduction to latex (or maybe not an introduction at all). Its purpose is to mostly introduce some of the symbols and notation used throughout the course, if you already know some of the basics.

\subsection*{Exercise 1}

\paragraph{(a)} 
My solution to this exercise is a bunch of \textbf{logical symbols}:
$\vee \wedge \Rightarrow \Leftrightarrow \neg \equiv \not\equiv \therefore$\\
And some more: $\exists_x \forall_y$

\paragraph{(b)}
Can't forget the \emph{set operations} and so forth:
\[ \cap \cup \setminus {}^\text{c} \mathcal{P} \mathcal{U} \in \not\in \subseteq \not\subseteq \emptyset \]

\paragraph{(c)}
Some basic mathematical notation
\[ \R \N \Z \Q < > \leq \geq \neq x^2 \sqrt{y} \]

\paragraph{(d)}

Here's a math table-like environment for setting up a series of derivations.
\begin{eqnarray*}
  && T \\
  &\equiv& \{ \text{rule to be proven } \Rightarrow \} \\
  && ? \\
  &\equiv& \{ \text{rule to be proven } \wedge \} \\
  && F 
\end{eqnarray*}

But if you don't want to use (a lot of) math inside a table, a normal table also works:\\
\begin{tabular}{lcr}
left & center & right \\
\hline
l & c & r \\
$x+y$ & $\equiv$ & $y + x$ \\
\end{tabular}

\begin{tabular}{rl}
& $a \wedge b$ \\
& $q$ \\
\hline
$\therefore$ & x 
\end{tabular}

\subsection*{Exercise 2}

\paragraph{(a)} Regular expressions: $(a + \varepsilon)^* (a+b) ab$
\paragraph{(a)} Useful for writing about Turing machines: $\square$

\subsection*{Exercise 3}

For specifying algorithms:

\begin{algorithm}{\sc MyIncredibleAlgorithm}[A,v]{
    \qinput an array $A$ of $n$ numbers and a number $v$
    \qoutput an index $i$ such that $A[i] = 42$, or \textsc{NotFound} if no such index exists}
    $i$ \qlet $1$ \\
    \qwhile $i \leq n$ and $A[i] \neq 42$ \\
     \qdo $i$ \qlet $i + v^2$ \qend \\
    \qif $i > n$ \\
    \qthen \qreturn \textsc{NotFound}\\
    \qelse \qreturn $i$ \qfi\\
    \qfor $i$ \qlet 1 to $n$ \\
    \qdo Something useful \qend
\end{algorithm}

For analysis, you may want to know $O, \Theta, \Omega(n \log n)$ (not to be confused with $o, \omega$).

\end{document}
